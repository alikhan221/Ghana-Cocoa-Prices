\chapter{Literature Review}

\section{Modeling Annual Cocoa Production Using ARIMA Time Series Model}

Time series forecasting methods such as ARIMA and Exponential Smoothing have long been recognized for their effectiveness in modeling agricultural commodity prices. These models are particularly valuable due to their ability to capture key structural components of time series data—trend, seasonality, and autocorrelation. Several studies have laid the foundation for using these methods in forecasting cocoa prices and production trends. \\

In their study, \textbf{Oni et al. (2021)} applied an ARIMA model to forecast cocoa production in Nigeria, identifying ARIMA(1,1,1) as the optimal specification. Their approach followed a rigorous statistical process: testing for stationarity using the Augmented Dickey-Fuller (ADF) test, examining ACF and PACF plots to guide model structure, and comparing model performance using MAE and RMSE. This study confirmed ARIMA’s strength in capturing agricultural seasonality and trend. \\

Meanwhile, \textbf{Kamu et al. (2010)} compared several univariate time series models, including ARIMA, Exponential Smoothing, and GARCH models, using monthly cocoa price data. Their analysis highlighted ARIMA's robustness in modeling average trends, but GARCH models performed better in environments of high volatility. Notably, exponential smoothing (ETS) provided relatively strong forecasts when seasonality was stable. \\

These foundational studies supported our decision to explore both \textbf{ARIMA} and \textbf{ETS} models. However, unlike Oni and Kamu, we expanded the analysis by introducing \textbf{exogenous variables} such as climate indicators and macroeconomic indices. This was inspired by recent institutional insights, including \textbf{J.P. Morgan’s 2024 report}, which emphasized how climate change and underinvestment in West African cocoa farms are exacerbating long-term supply shortages, thereby driving up prices. This real-world hypothesis encouraged the inclusion of \textbf{climate variables} (e.g., rainfall and temperature) in our models to explore potential lagged effects of environmental shocks. To determine the most effective model, we trained a set of baseline models including \textbf{ETS}, \textbf{ARIMA(0,1,1)}, \textbf{ARIMA(1,1,0)}, \textbf{ARIMA(1,1,1)}, \textbf{ARIMA(2,1,1)}, \textbf{ARIMA(3,1,0)}, and \textbf{auto.arima}, on monthly cocoa price data. Each model was evaluated on a 12-month test set using RMSE, MAE, AIC, and residual diagnostics such as the Ljung-Box test. Out of these models, \textbf{ARIMA(3,1,0)} provided the \textbf{lowest RMSE and MAE values} and the \textbf{cleanest residual diagnostics}, confirming its superior fit over the holdout period. Additionally, its simplicity (low pdq values) aligned with findings in the literature advocating for models with fewer than five total parameters to avoid overfitting while retaining forecast power (Kamu et al., 2010). This model also exhibited white-noise residuals and passed the Ljung-Box test, further validating its adequacy. \\

Given these results, we adopted ARIMA(3,1,0) as the primary forecasting model and used it as the benchmark for comparison with ARIMAX models incorporating external climate and economic variables. These models were tested with both contemporaneous and lagged predictors, allowing us to explore the temporal structure of supply and demand shocks.

\section{Summary}
While these studies established the value of ARIMA-based models in this space, our project takes a broader approach by building and comparing three different forecasting models: a manually tuned ARIMA model, a baseline model, and a climate-informed model. Unlike past work, which generally focuses only on historical price data, we incorporated climate indicators into our analysis to test more current and real-world hypotheses—specifically, the suggestion from J.P. Morgan (2024) that climate change and underinvestment in West African cocoa farms are driving long-term price increases. Given that West Africa contributes roughly 80\% of the world’s cocoa, supply-side issues such as poor yields and disease pressure have a massive effect on global prices. Our climate model aims to directly test these claims by incorporating environmental and supply-side variables, offering a more holistic and forward-looking view of price behavior. This makes our approach more adaptable to current market dynamics than traditional univariate time series models.


