\chapter{Introduction}

Cocoa is one of the most widely consumed agricultural commodities and forms the foundation of the multibillion-dollar chocolate industry. With the majority of cocoa production concentrated in West Africa—particularly in Ghana and Côte d’Ivoire—global cocoa prices are highly sensitive to regional environmental, political, and economic factors. These dependencies make cocoa markets highly volatile, and forecasting price movements presents a complex yet economically significant challenge.

This study applies time series analysis to forecast cocoa prices, with a focus on ARIMA modeling. Using historical monthly price data, we examine key patterns such as trends, seasonality, and autocorrelation to develop models capable of producing reliable short-term forecasts. In addition to a baseline ARIMA model, we construct two extended models: a \textbf{climate-informed model} incorporating Ghanaian temperature and precipitation data, and an \textbf{economic model} that includes U.S. macroeconomic indicators like the Consumer Price Index (CPI) and Producer Price Index (PPI). This multi-model framework allows us to test whether external climate and economic variables enhance forecasting performance.

After evaluating various ARIMA configurations using AIC and out-of-sample forecast accuracy, \textbf{ARIMA(3,1,0)} emerged as the most suitable baseline model. While climate and economic variables demonstrated some predictive value, particularly in lagged form, the core ARIMA model provided robust performance.

This project is motivated by the real-world importance of accurate commodity price forecasting. Stakeholders across the cocoa supply chain—farmers, exporters, manufacturers, and policymakers—depend on timely and reliable forecasts for production planning, pricing strategy, and risk management. By integrating econometric rigor with external climate and macroeconomic data, this study aims to contribute a practical and data-driven tool for understanding cocoa price behavior.



